\PassOptionsToPackage{unicode=true}{hyperref} % options for packages loaded elsewhere
\PassOptionsToPackage{hyphens}{url}
%
\documentclass[]{article}
\usepackage{lmodern}
\usepackage{amssymb,amsmath}
\usepackage{ifxetex,ifluatex}
\usepackage{fixltx2e} % provides \textsubscript
\ifnum 0\ifxetex 1\fi\ifluatex 1\fi=0 % if pdftex
  \usepackage[T1]{fontenc}
  \usepackage[utf8]{inputenc}
  \usepackage{textcomp} % provides euro and other symbols
\else % if luatex or xelatex
  \usepackage{unicode-math}
  \defaultfontfeatures{Ligatures=TeX,Scale=MatchLowercase}
\fi
% use upquote if available, for straight quotes in verbatim environments
\IfFileExists{upquote.sty}{\usepackage{upquote}}{}
% use microtype if available
\IfFileExists{microtype.sty}{%
\usepackage[]{microtype}
\UseMicrotypeSet[protrusion]{basicmath} % disable protrusion for tt fonts
}{}
\IfFileExists{parskip.sty}{%
\usepackage{parskip}
}{% else
\setlength{\parindent}{0pt}
\setlength{\parskip}{6pt plus 2pt minus 1pt}
}
\usepackage{hyperref}
\hypersetup{
            pdftitle={CTUHelpR - a vignette},
            pdfauthor={John Smith},
            pdfborder={0 0 0},
            breaklinks=true}
\urlstyle{same}  % don't use monospace font for urls
\usepackage[margin=1in]{geometry}
\usepackage{color}
\usepackage{fancyvrb}
\newcommand{\VerbBar}{|}
\newcommand{\VERB}{\Verb[commandchars=\\\{\}]}
\DefineVerbatimEnvironment{Highlighting}{Verbatim}{commandchars=\\\{\}}
% Add ',fontsize=\small' for more characters per line
\usepackage{framed}
\definecolor{shadecolor}{RGB}{248,248,248}
\newenvironment{Shaded}{\begin{snugshade}}{\end{snugshade}}
\newcommand{\AlertTok}[1]{\textcolor[rgb]{0.94,0.16,0.16}{#1}}
\newcommand{\AnnotationTok}[1]{\textcolor[rgb]{0.56,0.35,0.01}{\textbf{\textit{#1}}}}
\newcommand{\AttributeTok}[1]{\textcolor[rgb]{0.77,0.63,0.00}{#1}}
\newcommand{\BaseNTok}[1]{\textcolor[rgb]{0.00,0.00,0.81}{#1}}
\newcommand{\BuiltInTok}[1]{#1}
\newcommand{\CharTok}[1]{\textcolor[rgb]{0.31,0.60,0.02}{#1}}
\newcommand{\CommentTok}[1]{\textcolor[rgb]{0.56,0.35,0.01}{\textit{#1}}}
\newcommand{\CommentVarTok}[1]{\textcolor[rgb]{0.56,0.35,0.01}{\textbf{\textit{#1}}}}
\newcommand{\ConstantTok}[1]{\textcolor[rgb]{0.00,0.00,0.00}{#1}}
\newcommand{\ControlFlowTok}[1]{\textcolor[rgb]{0.13,0.29,0.53}{\textbf{#1}}}
\newcommand{\DataTypeTok}[1]{\textcolor[rgb]{0.13,0.29,0.53}{#1}}
\newcommand{\DecValTok}[1]{\textcolor[rgb]{0.00,0.00,0.81}{#1}}
\newcommand{\DocumentationTok}[1]{\textcolor[rgb]{0.56,0.35,0.01}{\textbf{\textit{#1}}}}
\newcommand{\ErrorTok}[1]{\textcolor[rgb]{0.64,0.00,0.00}{\textbf{#1}}}
\newcommand{\ExtensionTok}[1]{#1}
\newcommand{\FloatTok}[1]{\textcolor[rgb]{0.00,0.00,0.81}{#1}}
\newcommand{\FunctionTok}[1]{\textcolor[rgb]{0.00,0.00,0.00}{#1}}
\newcommand{\ImportTok}[1]{#1}
\newcommand{\InformationTok}[1]{\textcolor[rgb]{0.56,0.35,0.01}{\textbf{\textit{#1}}}}
\newcommand{\KeywordTok}[1]{\textcolor[rgb]{0.13,0.29,0.53}{\textbf{#1}}}
\newcommand{\NormalTok}[1]{#1}
\newcommand{\OperatorTok}[1]{\textcolor[rgb]{0.81,0.36,0.00}{\textbf{#1}}}
\newcommand{\OtherTok}[1]{\textcolor[rgb]{0.56,0.35,0.01}{#1}}
\newcommand{\PreprocessorTok}[1]{\textcolor[rgb]{0.56,0.35,0.01}{\textit{#1}}}
\newcommand{\RegionMarkerTok}[1]{#1}
\newcommand{\SpecialCharTok}[1]{\textcolor[rgb]{0.00,0.00,0.00}{#1}}
\newcommand{\SpecialStringTok}[1]{\textcolor[rgb]{0.31,0.60,0.02}{#1}}
\newcommand{\StringTok}[1]{\textcolor[rgb]{0.31,0.60,0.02}{#1}}
\newcommand{\VariableTok}[1]{\textcolor[rgb]{0.00,0.00,0.00}{#1}}
\newcommand{\VerbatimStringTok}[1]{\textcolor[rgb]{0.31,0.60,0.02}{#1}}
\newcommand{\WarningTok}[1]{\textcolor[rgb]{0.56,0.35,0.01}{\textbf{\textit{#1}}}}
\usepackage{graphicx,grffile}
\makeatletter
\def\maxwidth{\ifdim\Gin@nat@width>\linewidth\linewidth\else\Gin@nat@width\fi}
\def\maxheight{\ifdim\Gin@nat@height>\textheight\textheight\else\Gin@nat@height\fi}
\makeatother
% Scale images if necessary, so that they will not overflow the page
% margins by default, and it is still possible to overwrite the defaults
% using explicit options in \includegraphics[width, height, ...]{}
\setkeys{Gin}{width=\maxwidth,height=\maxheight,keepaspectratio}
\setlength{\emergencystretch}{3em}  % prevent overfull lines
\providecommand{\tightlist}{%
  \setlength{\itemsep}{0pt}\setlength{\parskip}{0pt}}
\setcounter{secnumdepth}{0}
% Redefines (sub)paragraphs to behave more like sections
\ifx\paragraph\undefined\else
\let\oldparagraph\paragraph
\renewcommand{\paragraph}[1]{\oldparagraph{#1}\mbox{}}
\fi
\ifx\subparagraph\undefined\else
\let\oldsubparagraph\subparagraph
\renewcommand{\subparagraph}[1]{\oldsubparagraph{#1}\mbox{}}
\fi

% set default figure placement to htbp
\makeatletter
\def\fps@figure{htbp}
\makeatother


\title{CTUHelpR - a vignette}
\author{John Smith}
\date{2020-03-31}

\begin{document}
\maketitle

{
\setcounter{tocdepth}{3}
\tableofcontents
}
\newpage

\hypertarget{introduction}{%
\section{Introduction}\label{introduction}}

This R package does something useful\ldots{} for sure\ldots{}

\hypertarget{installing-from-github-with-devtools}{%
\section{\texorpdfstring{Installing from GitHub with
\texttt{devtools}}{Installing from GitHub with devtools}}\label{installing-from-github-with-devtools}}

Let's get started by installing the package straight from
\href{https://github.com/SwissClinicalTrialOrganisation/secuTrialR}{\textcolor{blue}{GitHub}}
and then loading it. For this you will need to have \texttt{devtools}
installed. We are planning to add \texttt{secuTrialR} to CRAN but we are
not there yet. If you are working on Windows and would like to install
with \texttt{devtools} you will likely need to install
\href{https://cran.r-project.org/bin/windows/Rtools/}{\textcolor{blue}{Rtools}}.

\vspace{5pt}

\begin{Shaded}
\begin{Highlighting}[]
\CommentTok{# install}
\NormalTok{devtools}\OperatorTok{::}\KeywordTok{install_github}\NormalTok{(}\StringTok{"CTU-Basel/CTUHelpR"}\NormalTok{)}
\end{Highlighting}
\end{Shaded}

\hypertarget{loading-the-package}{%
\section{Loading the package}\label{loading-the-package}}

\begin{Shaded}
\begin{Highlighting}[]
\CommentTok{# load}
\KeywordTok{library}\NormalTok{(CTUHelpR)}
\CommentTok{# show CTUHelpR version}
\KeywordTok{installed.packages}\NormalTok{()[}\StringTok{"CTUHelpR"}\NormalTok{, }\StringTok{"Version"}\NormalTok{]}
\CommentTok{#> [1] "0.0.1"}
\end{Highlighting}
\end{Shaded}

\hypertarget{printing-the-contents-of-a-text-file}{%
\section{Printing the contents of a text
file}\label{printing-the-contents-of-a-text-file}}

\begin{Shaded}
\begin{Highlighting}[]
\NormalTok{path <-}\StringTok{ }\KeywordTok{system.file}\NormalTok{(}\StringTok{"exdata"}\NormalTok{, }\StringTok{"file.txt"}\NormalTok{,}
                    \DataTypeTok{package =} \StringTok{"CTUHelpR"}\NormalTok{)}
\KeywordTok{print_file_content}\NormalTok{(}\DataTypeTok{file_path =}\NormalTok{ path)}
\CommentTok{#> Hello world!}
\end{Highlighting}
\end{Shaded}

\newpage

\begin{Shaded}
\begin{Highlighting}[]
\KeywordTok{sessionInfo}\NormalTok{()}
\CommentTok{#> R version 3.6.3 (2020-02-29)}
\CommentTok{#> Platform: x86_64-pc-linux-gnu (64-bit)}
\CommentTok{#> Running under: Ubuntu 18.04.4 LTS}
\CommentTok{#> }
\CommentTok{#> Matrix products: default}
\CommentTok{#> BLAS:   /usr/lib/x86_64-linux-gnu/blas/libblas.so.3.7.1}
\CommentTok{#> LAPACK: /usr/lib/x86_64-linux-gnu/lapack/liblapack.so.3.7.1}
\CommentTok{#> }
\CommentTok{#> locale:}
\CommentTok{#>  [1] LC_CTYPE=C.UTF-8       LC_NUMERIC=C           LC_TIME=C.UTF-8       }
\CommentTok{#>  [4] LC_COLLATE=C.UTF-8     LC_MONETARY=C.UTF-8    LC_MESSAGES=C.UTF-8   }
\CommentTok{#>  [7] LC_PAPER=C.UTF-8       LC_NAME=C              LC_ADDRESS=C          }
\CommentTok{#> [10] LC_TELEPHONE=C         LC_MEASUREMENT=C.UTF-8 LC_IDENTIFICATION=C   }
\CommentTok{#> }
\CommentTok{#> attached base packages:}
\CommentTok{#> [1] stats     graphics  grDevices utils     datasets  methods   base     }
\CommentTok{#> }
\CommentTok{#> other attached packages:}
\CommentTok{#> [1] CTUHelpR_0.0.1 rmarkdown_2.1 }
\CommentTok{#> }
\CommentTok{#> loaded via a namespace (and not attached):}
\CommentTok{#>  [1] Rcpp_1.0.3      crayon_1.3.4    zeallot_0.1.0   packrat_0.5.0  }
\CommentTok{#>  [5] digest_0.6.22   R6_2.4.0        backports_1.1.5 magrittr_1.5   }
\CommentTok{#>  [9] evaluate_0.14   pillar_1.4.2    rlang_0.4.1     stringi_1.4.3  }
\CommentTok{#> [13] vctrs_0.2.0     tools_3.6.3     stringr_1.4.0   readr_1.3.1    }
\CommentTok{#> [17] hms_0.5.2       xfun_0.10       yaml_2.2.0      compiler_3.6.3 }
\CommentTok{#> [21] pkgconfig_2.0.3 htmltools_0.4.0 knitr_1.25      tibble_2.1.3}
\end{Highlighting}
\end{Shaded}

\end{document}
